\documentclass[11pt,a4paper,sans]{moderncv}
\moderncvstyle{banking}
\moderncvcolor{black}
\nopagenumbers{}
\usepackage[utf8]{inputenc}
\usepackage{ragged2e}
\usepackage[
    a4paper,
    left=5mm,
    right=5mm,
    top=5mm,
    bottom=0mm]{geometry}
\usepackage{import}
\usepackage{multicol}
\usepackage{import}
\usepackage{enumitem}
\usepackage{amssymb}
\usepackage{lmodern}

\name{Cristóbal Andrés}{Contreras Beltrán}

\newcommand{\sectionMargin}{-3mm}
\newcommand*{\customcventry}[7][.13em]{
    \begin{tabular}{@{}l}
    {\bfseries #4} \
    {\itshape #3}
    \end{tabular}
    \hfill
    \begin{tabular}{l@{}}
    {\bfseries #5} \
    {\itshape #2}
    \end{tabular}
    \ifx&#7&%
    \else{\
    \begin{minipage}{\maincolumnwidth}%
    \small#7%
    \end{minipage}}\fi%
    \par\addvspace{#1}
}

\begin{document}

\makecvtitle
\vspace*{-11mm}
\begin{center}
    \textbf{Ingeniero civil Informático - Soporte Técnico}
\end{center}

\vspace*{-7mm}

\begin{center}
    \begin{tabular}{ c @{\hskip 1em} c @{\hskip 1em} c }
        \faMobile \enspace +56 9 9518 2311
        &
        \faEnvelope \enspace cristobal.contreras@outlook.cl
        &
        \faHome \enspace Temuco, Chile
    \\
        \faLinkedin\enspace
        \href{https://www.linkedin.com/in/cristobal-contreras-beltran/}{\underline{/in/cristobal-contreras}}
        &
        \faGithub\enspace
        \href{https://www.github.com/AsCraftC}{\underline{/AsCraftC}}
        &
        \faBehance\enspace
        \href{https://www.behance.net/AsCraftC}{\underline{/AsCraftC}}
    \end{tabular}
\end{center}

\vspace*{-10mm}

\section{Perfil}{
    Ingeniero recién egresado con bases de muchas ramas de la informática y dirección de proyectos. Gran capacidad de aprender tecnologías de desarrollo de software, base de datos y soporte técnico informático. Con experiencia en soporte técnico enfocado al usuario. Especializado en \textbf{Frontend}, experiencia de usuario \textbf{(UX)}, manejo de base de datos \textbf{SQL y NoSQL} y conocimientos de \textbf{Backend}.
}

\vspace*{\sectionMargin}

\section{Areas de Competencia}{
    Diseño y mantención de base de datos - Administrador Windows - GNU/Linux - Soporte Técnico - Diseño UX/UI - Desarrollo Frontend - Gamificación - Diseño de juegos - Administración de redes - CPS.
}

\vspace*{\sectionMargin}

\section{Experiencia Profesional}{
    \customcventry
        {09/2024 ‐ 10/2024}
        {{ \href{https://www.araucaniasur.cl/index.php/hospital-de-villarrica/}{\underline{Hospital de Villarrica}} }}
        {Soporte Técnico,}
        {Villarrica, Chile}{}
        {{\begin{itemize}[leftmargin=0.6cm, noitemsep, label={\textbullet}]
            \item Configuración y reparación de equipos institucionales, optimizando la experiencia del usuario y simplificando tareas cotidianas.
            \item Configuración de servicios, software y redes de oficinas, con un enfoque en la comodidad y eficiencia de los funcionarios.
            \item Implementación de estrategias de respuesta ágil para la resolución inmediata de problemas en el lugar de operación.
            \item Administración avanzada, configuración y reparación de Sistemas ofimáticos Operativos Windows.
        \end{itemize}}
    }

    \customcventry
        {06/2023 ‐ 12/2023}
        {{ \href{https://ideaufro.com/}{\underline{iDEAUFRO}} }}
        {Desarrollador Frontend,}
        {Temuco, Chile}{}
        {{\begin{itemize}[leftmargin=0.6cm, noitemsep, label={\textbullet}]
            \item Levantamiento de requerimientos de proyecto y necesidades relacionadas al Frontend.
            \item Uso de Figma para la creación de wireframes y mockups interactivos, desarrollando un total de 7 vistas.
            \item Diseño y desarrollo de interfaces web aplicando diseño centrado en el usuario en 4 iteraciones.
            \item Integración con el backend mediante llamadas a la API y despliegue en servidor utilizando Docker.
        \end{itemize}}
    }

    \vspace*{2mm}

    \customcventry
        {02/2021 ‐ 04/2024}
        {{ \href{http://www.tisol.cl/}{\underline{ASESORIAS Y GESTIONES TECNOLOGICAS SPA}} }}
        {Soporte Técnico,}
        {La Araucania, Chile}{}
        {{\begin{itemize}[leftmargin=0.6cm, noitemsep, label={\textbullet}]
            \item Diagnóstico y reparación de impresoras en problemas de hardware y/o software.
            \item Configuración y validación de impresoras en redes LAN.
            \item Configuración de usabilidad de equipos de impresión y capacitación de usuarios en su uso.
            \item Optimización del uso de equipos de impresión, aumentando la productividad de los usuarios y reduciendo la necesidad de intervención técnica.
        \end{itemize}}
    }
}

\vspace*{\sectionMargin}

\section{Cursos En linea \& Certificaciones}{
    \begin{itemize}[label=\textbullet]
        \item Ruta de Ciberseguridad Personal (May. 2024) \href{https://1drv.ms/b/c/13c8ae619d64655e/EZYaMe6SBhJAshtvq4ORCQoBwWMYRerI4_xiuqXjORVd0w?e=I4tLqg}{\underline{Platzi}}
        \item Curso profesional de Git y GitHub (Jun. 2024) \href{https://1drv.ms/b/c/13c8ae619d64655e/EXeQGPrTCjFDpmx8pCItVTwB-OS1r-tkKJbmgYnahuxtYg?e=mc9YcH}{\underline{Platzi}}
    %   \item Aprende Unity y C\# desarrollando juegos reales (Mayo. 2024) - \underline{\color{blue}\href{https://www.udemy.com/}{Udemy}}
    %   \item Power BI TOTAL en 14 Dias - Analista de datos avanzado (May. 2024) - \underline{\color{blue}\href{https://www.udemy.com/}{Udemy}}
    \end{itemize}
}

\vspace*{-6mm}

\section{Educación}{
    \customcventry
        {2018-2023}
        { \href{https://1drv.ms/b/c/13c8ae619d64655e/EV5lZJ1hrsgggBOHLAAAAAABHh9OZUbWLd2nx7PTa_OSsg?e=Fedmge}
        {\underline{Universidad de La Frontera}} }
        {Ingeniería Civil Informática}
        {Temuco, Chile}{}{}
        {Cursos Relevantes: 
            {Taller de redes}, 
            {Ciencias de la Computación} 1 y 2, 
            \textbf{Base de datos 1 y 2}, 
            \textbf{Ingeniería de requerimientos},
            {Auditoria Informática}, 
            {Diseño de experiencia de usuario}, 
            {Criptografía y privacidad}, 
            Gestión de {calidad del software}, 
            \textbf{Formulación y evaluación de proyectos},
            {Seguridad de la Información},
            Internet de la cosas{(IoT)}
    }
}

\vspace*{\sectionMargin}
%       \textbf
\section{Habilidades}{
    \begin{itemize}[label=\textbullet]
        \item {\underline{\textbf{Base de Datos:}}
            \textbf{MySQL},
            \textbf{PostgreSQL},
            {MongoDB},
            {MariaDB}
        }
        \item {\underline{\textbf{Lenguajes de programación y otros:}}
            \textbf{HTML/CSS/JavaScript},
            {Java},
            \textbf{SQL},
            \textbf{Python},
            {C\#},
            {GDScript},
            {Arduino}
            {LateX}
        }
        \item {\underline{\textbf{Software y Frameworks:}}
            {React},
            {Vue3},
            {SASS},
            {Springboot},
            {Flutter},
            {Unity},
            {Godot},
            \textbf{Word},
            \textbf{PowerPoint},
            \textbf{Excel}, 
            {Figma},
            {npm},
            {git},
            {git flow},
            {GitHub},
            {VSCode},
            {Android Studio},
            {IntelliJ IDEA},
            {Windows} (7, 10 y 11),
            {GNU/Linux} (Ubuntu, CentOS, ArchLinux, fedora, kali)
        }
        \item {\textbf{Habilidades adicionales:} 
            Consumo y creación de \textbf{API REST},
            \textbf{Documentación IEE-830},
            {UML},
            {Prueba Unitarias},
            \textbf{Diseño responsivo},
            Evaluación y {Manejo de riesgos},  
            Autogestión, 
            Licencia de conducir Clase B
        }
    \end{itemize}
}

\vspace*{\sectionMargin}

\section{Idiomas}{
    \begin{multicols}{2}
        \begin{itemize}[label=\textbullet]
        \item \textbf{Español} [Nativo]
        \item \textbf{Inglés - Conversacional} [Intermedio] - B1 (UFRO)
        \item \textbf{Inglés - Lectura/Escritura} [Técnico]
        \end{itemize}
    \end{multicols}
}

\end{document}