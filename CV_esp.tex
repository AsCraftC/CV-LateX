\documentclass[11pt,a4paper,sans]{moderncv}
\moderncvstyle{banking}
\moderncvcolor{black}
\nopagenumbers{}
\usepackage[utf8]{inputenc}
\usepackage[
    a4paper,
    left=5mm,
    right=5mm,
    top=5mm,
    bottom=0mm]{geometry}
\usepackage{enumitem}
\usepackage{lmodern}

\name{Cristóbal}{Contreras Beltrán}

\newcommand{\sectionMargin}{-3mm}
\newcommand*{\customcventry}[7][.13em]{
    \begin{tabular}{@{}l}
    {\bfseries #4} \
    {\itshape #3}
    \end{tabular}
    \hfill
    \begin{tabular}{l@{}}
    {\bfseries #5} \
    {\itshape #2}
    \end{tabular}
    \ifx&#7&%
    \else{\
    \begin{minipage}{\maincolumnwidth}%
    \small#7%
    \end{minipage}}\fi%
    \par\addvspace{#1}
}

\begin{document}

\makecvtitle
\vspace*{-11mm}
\begin{center}
    \textbf{Ingeniero de Software - Desarrollador Frontend}
\end{center}

\vspace*{-7mm}

\begin{center}
    \begin{tabular}{ c @{\hskip 1em} c @{\hskip 1em} c }
        \faMobile   \enspace +56 9 9518 2311
        &
        \faEnvelope \enspace cristobal.contreras@outlook.cl
        &
        \faHome     \enspace Labranza, Chile
    \\
        \faLinkedin\enspace
        \href{https://www.linkedin.com/in/cristobal-contreras-beltran/}{\underline{/in/cristobal-contreras}}
        &
        \faGithub\enspace
        \href{https://www.github.com/AsCraftC}{\underline{/AsCraftC}}
        &
        \faBehance\enspace
        \href{https://www.behance.net/AsCraftC}{\underline{/AsCraftC}}
    \end{tabular}
\end{center}

\vspace*{-10mm}

\section{Perfil}{
    Ingeniero Civil Informático con experiencia en \textbf{soporte técnico}, desarrollo \textbf{frontend}, Experiencia de usuario\textbf{(UX)}/Interfaces de usuario\textbf{(UI)} y Diseño centrado en el usuario\textbf{(UCD)}. Capacitado para implementar soluciones tecnológicas eficientes, desde configuración, administración y capacitación de sistemas hasta desarrollo de aplicaciones web. Conocimiento en bases de datos \textbf{SQL y NoSQL}, así como en la integración y desarrollo de \textbf{backend (API y microservicos)}. Enfocado en la mejora de procesos tecnológicos para maximizar la productividad y la satisfacción del usuario final.
}

\vspace*{\sectionMargin}

\section{Areas de Competencia}{
    Desarrollo Frontend - Diseño UX/UI - Gamificación - Diseño de juegos - Base de Datos - API REST - Soporte Técnico - Administrador Windows - GNU/Linux - Administración de redes - CPS.
}

\vspace*{\sectionMargin}

\section{Experiencia Profesional}{
\customcventry
    {09/2024 ‐ 12/2024}
    {{ \href{https://www.araucaniasur.cl/index.php/hospital-de-villarrica/}{\underline{Hospital de Villarrica}} }}
    {Soporte Técnico,}
    {Villarrica, Chile}{}
    {{\begin{itemize}[leftmargin=0.6cm, noitemsep, label={\textbullet}]
        \item Optimización de experiencia de usuario y tareas diarias mediante una configuración y reparación de equipos institucionales.
        \item Aumento de la comodidad y eficiencia de los empleados mediante la configuración de servicios, software y redes de oficina.
        \item Implementación de estrategias de respuesta ágil para la resolución inmediata de problemas en puntos operativos.
        \item Administración, configuración y reparación expertas de sistemas Windows para garantizar un funcionamiento sin problemas.
    \end{itemize}}
}
\vspace*{2mm}
\customcventry
    {06/2023 ‐ 12/2023}
    {{ \href{https://ideaufro.com/}{\underline{iDEAUFRO}} }}
    {Desarrollador Frontend,}
    {Temuco, Chile}{}
    {{\begin{itemize}[leftmargin=0.6cm, noitemsep, label={\textbullet}]
        \item Recopilación y análisis de los requisitos del proyecto frontend para asegurar la alineación con las necesidades del cliente.
        \item Creación de wireframes y maquetas interactivas con Figma, cubriendo un total de 7 vistas.
        \item Diseño y desarrollo de la interfaz web, aplicando los principios de diseño centrado en el usuario en 4 iteraciones.
        \item Integración de servicios backend utilizando APIs RESTful y despliegue de servidores con Docker.
    \end{itemize}}
}
\vspace*{2mm}
\customcventry
    {02/2021 ‐ 04/2024}
    {{ \href{http://www.tisol.cl/}{\underline{ASESORIAS Y GESTIONES TECNOLOGICAS SPA}} }}
    {Soporte Técnico,}
    {La Araucania, Chile}{}
    {{\begin{itemize}[leftmargin=0.6cm, noitemsep, label={\textbullet}]
        \item Diagnóstico y reparación de problemas de hardware y software de impresoras.
        \item Configuración y validación del uso de impresoras en redes LAN.
        \item Formación de usuarios en uso y mejores prácticas de equipos de impresión.
        \item Optimización del uso de impresoras, aumentando la productividad de los usuarios y reduciendo las necesidades de intervenciones técnicas.
    \end{itemize}}
}
}

\vspace*{\sectionMargin}

\section{Cursos En linea y Certificaciones}{
\begin{itemize}[label=\textbullet, noitemsep]
    \item Ruta de Ciberseguridad Personal (May. 2024) \href{https://1drv.ms/b/c/13c8ae619d64655e/EZYaMe6SBhJAshtvq4ORCQoBwWMYRerI4_xiuqXjORVd0w?e=I4tLqg}{\underline{Platzi}}
    \item Curso profesional de Git y GitHub (Jun. 2024) \href{https://1drv.ms/b/c/13c8ae619d64655e/EXeQGPrTCjFDpmx8pCItVTwB-OS1r-tkKJbmgYnahuxtYg?e=mc9YcH}{\underline{Platzi}}
%   \item Aprende Unity y C\# desarrollando juegos reales (Mayo. 2024) - \underline{\color{blue}\href{https://www.udemy.com/}{Udemy}}
%   \item Power BI TOTAL en 14 Dias - Analista de datos avanzado (May. 2024) - \underline{\color{blue}\href{https://www.udemy.com/}{Udemy}}
\end{itemize}
}

\vspace*{-6mm}

\section{Educación}{
\customcventry
    {2018-2023}
    { \href{https://1drv.ms/b/c/13c8ae619d64655e/EV5lZJ1hrsgggBOHLAAAAAABHh9OZUbWLd2nx7PTa_OSsg?e=Fedmge}{\underline{Universidad de La Frontera}} }
    {Ingeniería Civil Informática}
    {Temuco, Chile}
    {}{}
    {Cursos Relevantes: 
        {Taller de redes}, 
        {Ciencias de la Computación} 1 y 2, 
        \textbf{Base de datos 1 y 2}, 
        \textbf{Ingeniería de requerimientos},
        {Auditoria Informática}, 
        {Diseño de experiencia de usuario}, 
        {Criptografía y privacidad}, 
        Gestión de {calidad del software}, 
        \textbf{Formulación y evaluación de proyectos},
        {Seguridad de la Información},
        Internet de la cosas{(IoT)}
}
}

\vspace*{\sectionMargin}
%       \textbf
\section{Habilidades}{
    \begin{itemize}[label=\textbullet, noitemsep]
        \item {\underline{\textbf{Lenguajes de programación y otros:}}
            {HTML/CSS/JavaScript},
            {Java},
            {SQL},
            {Python},
            {C\#},
            {Dart},
            {GDScript},
            {Arduino},
            {LateX}
        }
        \item {\underline{\textbf{Base de Datos:}}
            {MySQL},
            {PostgreSQL},
            {MongoDB},
            {MariaDB}
        }
        \item {\underline{\textbf{Software y Frameworks:}}
            \textbf{React},
            \textbf{Astro}
            {Vue3},
            {SASS},
            {Springboot},
            {Flutter},
            {Unity},
            {Godot},
            {Word},
            {PowerPoint},
            {Excel}, 
            \textbf{Figma},
            \textbf{npm},
            \textbf{git},
            {git flow},
            {GitHub},
            {VSCode},
            {Android Studio},
            {IntelliJ IDEA},
            {Windows} (7, 10 y 11),
            {GNU/Linux} (Ubuntu, CentOS, ArchLinux, fedora, kali)
        }
        \item {\textbf{Habilidades adicionales:} 
            Consumo y creación de {API REST},
            {Documentación IEE-830},
            {UML},
            {Prueba Unitarias},
            {Diseño responsivo},
            Evaluación y {Manejo de riesgos},  
            {Autogestión}, 
            Licencia de conducir {Clase B}
        }
    \end{itemize}
}

\vspace*{\sectionMargin}

\section{Idiomas}{
\begin{tabbing}
    \texttt{drwx}\hspace{1mm}\= \textbf{Español}    \hspace{1mm}\=  [Nativo] \\
    \texttt{d----x} \>          \textbf{Inglés}     \>              [B1] \\
    \texttt{drw--}  \>          \textbf{Inglés}     \>              [Técnico]
\end{tabbing}

}

\end{document}